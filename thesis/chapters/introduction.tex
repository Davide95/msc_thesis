\chapter{Introduction and motivation}

In the field of biology, unsupervised methods like ARACNE (see \cite{DBLP:journals/bmcbi/MargolinNBWSFC06}) 
are used to obtain Gene Regulatory Networks starting from microarray expression profiles. 

Methods like Relational Topic Model (introduced in \cite{pmlr-v5-chang09a}) were also proposed to 
model a collection of documents and the links between them in a supervised way. 

The goal of this thesis is to propose a novel approach to perform network inference given a collection of 
webpages scraped from a website, taking the best from the two methods cited before. 
In particular, the final result is an unweighted and undirected graph which connects pages with similar content.

The manuscript is organized as follows: 
in Chapter \ref{wordemb} we present some techniques to represent words numerically. 
In Chapter \ref{topicmod} we move from word to document representations and we describe Topic Modeling. 
In Chapter \ref{networks} we describe network inference, presenting ARACNE.
In Chapter \ref{experiments} and Chapter \ref{conclusions} we explain step-by-step our original work 
and we propose some future directions. 
Finally, Appendices \ref{ann}, \ref{pdib}, \ref{sdist} and \ref{vi} cover a part of the background knowledge needed to read this document.