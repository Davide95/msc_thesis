\chapter{Source code}
All source code is licensed under the GNU General Public License v3.0 and it can be found at \url{https://github.com/Davide95/msc_thesis/tree/master/code}.
In order to have a self-contained document, the realized scripts are also inserted in this appendix.
You can find a copy of the license at \url{https://github.com/Davide95/msc_thesis/blob/master/code/LICENSE}.

All code is written in Python 3.7. The external libraries used are reported in the relative \texttt{requirements.txt} file and can be installed through the command \texttt{pip install -r requirements.txt}.

A set of tests that can be found at \url{https://github.com/Davide95/msc_thesis/tree/master/tests} were done to check the validity of the code.
The source code was also analyzed with \say{flake8} and \say{pylint}.

All experiments were tested on a PowerEdge T630 server with the following characteristics:
\begin{itemize}
    \item Intel(R) Xeon(R) CPU E5-2630 v3 @ 2.40GHz (2 sockets)
    \item 8x16GiB of RAM
\end{itemize}

\section{Requirements}
\begin{minipage}{\linewidth}
    \inputminted{text}{../code/requirements.txt}
\end{minipage}

\pagebreak
\section{Spider} \label{spider}

Since the spider implemented is built on top of the web-crawling platform Scrapy, we refer to its documentation instead of explaining how to run it.
An overview of Scrapy can be found in \cite{kouzis2016learning}.

\inputminted{Python}{../code/custom_spider.py}

\pagebreak
\section{Random spider} \label{randomspider}

Since the spider implemented is built on top of the web-crawling platform Scrapy, we refer to its documentation instead of explaining how to run it.
An overview of Scrapy can be found in \cite{kouzis2016learning}.

\inputminted{Python}{../code/random_spider.py}

\pagebreak
\section{Pre-processing} \label{preprocessing-code}
The script has some mandatory and optional parameters.
Run \say{python preprocessing.py --help} for more information.

\inputminted{Python}{../code/preprocessing.py}

\pagebreak
\section{Similarity graph} \label{simgra}
The script has some mandatory and optional parameters.
Run \say{python similarity\_graph.py --help} for more information.

If the machine in which you are running the code suffers from out of memory errors,
try to increase the value of the \texttt{--mul} parameter.
If it does not work, you have to decrease the value of the \texttt{--n\_jobs} parameter.

\inputminted{Python}{../code/similarity_graph.py}