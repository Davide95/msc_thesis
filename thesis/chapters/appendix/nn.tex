\chapter{Neural Networks}

\section{Gradient descent}
Given a minimization problem $min_w f(w), w \in \mathbb{R}^D$,
it is possible to find a local minimum through gradient descent.

First, we compute the gradient $\nabla f(w)$ and we decide a point $w_0$ in which the algorithm has to start.
If the problem is not convex, choosing a different $x_0$ could lead to a different
solution.
Since gradients point in the direction of fastest increase,
we take small steps towards the opposite direction in order to reach a minimum.

The step size is controlled by a hyperparameter $\lambda$:
values too small for a particular problem increase
the time needed to obtain a good solution and
it can lead to a local minimum;
values too big prevent convergence.

The final equation is:
$$ w_{i+1} = w_{i} - \lambda \nabla f(w_i) $$

Each step is run iteratively until a stopping criterion is met.
If the algorithm is stopped after $I$ iterations, the solution
for the minimization problem is $w_{I}$.
Common sense stopping criterions could be:
\begin{itemize}
    \item stop after a prefixed number of iterations
    \item stop when the results are not getting better for the last $I$ steps
    \item stop when $f(w_i) - f(w_{i+1}) < \epsilon$, with $\epsilon > 0$ small
\end{itemize}

Note that the parameter $\lambda$ could be different at each iteration:
$$ w_{i+1} = w_{i} - \lambda_{i+1} \nabla f(w_i) $$
For instance, we could have $\lambda_i = \frac{\lambda}{i}$ in order to take
smaller and smaller steps until convergence.

\subsection{Gradient descent in action} \label{gradesc}
Suppose to have a dataset $X \in \mathbb{R}^{N \times D}, y \in \mathbb{R}^N$.

What we want to do is linear regression, whose model is obtained resolving:
$$\displaystyle \min_{w \in \mathbb{R}^D} \frac{1}{2N} \sum_i (w^T x_i - y_i)^2$$

The derivative  $\frac{1}{2N} \sum_i (w^T x_i - y_i)^2 dw$ is:
$$\frac{1}{2N} \sum_i 2 x_i(w^T x_i - y_i) = \frac{\sum_i x_i(w^T x_i - y_i)}{N}$$

Therefore:
$$ w_{n+1} = w_n - \frac{\lambda}{N} \sum_i x_i(w^T x_i - y_i) $$


\section{Mini-batch gradient descent}
There are situations in which using gradient descent is not computationally feasible.
For instance, imagine that the problem explained in Section \ref{gradesc} have
a number of points so huge that a solution can not be obtained in feasible times.

Mini-batch gradient descent is an extension of gradient descent which assume that
samples of $|M|$ elements of the dataset are a noisy representative of the entire data.
At each step, $M$ will be re-sampled from $X$.

In that case, the minimization presented in Section \ref{gradesc} can be rewritten as:
$$\displaystyle \min_{w \in \mathbb{R}^D} \frac{1}{2|M|} \sum_{x_i \in M} (w^T x_i - y_i)^2 $$