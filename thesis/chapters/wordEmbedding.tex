\chapter{Word Embedding} \label{wordemb}
Documents are made by words.
To be manipulable by an algorithm these words need to be represented in a numerical format.

This chapter covers some techniques able to fulfil this task,
pointing out the advantages and disadvantages of choosing one instead of another.

To avoid misunderstandings
we define the term \say{word} as a word in a particular position in a document,
while we use the word \say{term} as a generic word into a dictionary.

For instance, given a document \say{A cat is an animal but an animal is not a cat} its list of words
is $ w = (a, cat, is, an, animal, but, an, animal, is, not, a, cat)$
while its set of terms is $d = \{a, cat, is, an, animal, but, not\}$.

\section{One Hot Encoding}
Given a list of words $w = (w_1, w_2, \dots, w_N)$,
we first generate a dictionary of unique words
$d = \{d_1, d_2, ..., d_K\}$ with an arbitrary order between words.
Note that $ K \leq N $ is always true and $ K = N $ happens only when there is no repetition of words in the documents.

Each term $d_i$ in the dictionary can now be encoded as $x_i \in \{0,1\}^K$ following the One Hot Encoding approach.
All elements of $x_i$ are zero except a \say{1} in the position  $i$ of the vector.
That means that two words $w_l, w_j$ in different positions will have the same value $x_i$ if and only if they correspond to the same term $d_i$.

Applying One Hot Encoding on the previous example presented in the introduction of this chapter leads to the following values of $x_i$:
\begin{multicols}{2}
    \begin{itemize}
        \item $x_1 = x_{a} = [1, 0, 0, 0, 0, 0, 0]$
        \item $x_2 = x_{cat} = [0, 1, 0, 0, 0, 0, 0]$
        \item $x_3 = x_{is} = [0, 0, 1, 0, 0, 0, 0]$
        \item $x_4 = x_{an} = [0, 0, 0, 1, 0, 0, 0]$
        \item $x_5 = x_{animal} = [0, 0, 0, 0, 1, 0, 0]$
        \item $x_6 = x_{but} = [0, 0, 0, 0, 0, 1, 0]$
        \item $x_7 = x_{not} = [0, 0, 0, 0, 0, 0, 1]$
    \end{itemize}
\end{multicols}

Despite its simplicity and the fact that it is not computationally expensive to use, it has some disadvantages:
\begin{itemize}
    \item vectors are sparse: only one element of each vector is not zero
    \item there is no concept of similarity: all vectors have the same distance between each other
\end{itemize}

\section{Word2vec}
\dots

\section{Global Vectors for Word Representation (GloVe)}
\dots