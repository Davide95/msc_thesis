\chapter{Experiments}
This chapter explains step-by-step our original work, built on top of the theory presented in the previous chapters.

Steps are grouped into sections, sorted by chronological order.
The last section (Section \ref{scaling}) considers instead the scalability issues that we encountered and how we managed them.



\section{Crawling}
In order to collect data, 
we implemented a spider that given a particular domain, it downloads and store all HTML documents 
and the relative structure built upon the links that we encounter.

The application can be summarized with these steps:
\begin{enumerate}
    \item first, the software starts from an URL defined by the user, putting it into a pool
    \item if the pool is not empty, the application will get a link from it, starting its download. After getting the link, it is removed from the pool
    \item if the content is a valid HTML document, it is stored
    \item all links of that webpage which are in the specified domain are stored. They are also put into the pool if they were not analyzed previously
    \item loop to step 2 until there are no links left
\end{enumerate}

The final result is a set of tuples \texttt{(url, connected\_to, content)}, where \texttt{url} is the URL of a particular page, \texttt{connected\_to} is its set of links and \texttt{content} is its HTML source code.
Scrapy allows saving these results in different formats, but we choose to save everything in CSV files in order to re-use them easily in the next phases.

Before using the scraped data, we performed a data preparation phase over it.
In this phase we make a strong assumption about webpages:
different URLs with the same scheme correspond to the same page.
For instance, \url{http://www.example.com} and \url{https://www.example.com}
are considered links to the same resource.

Following this assumtion, we then removed all duplicates from the \say{url} column
of the .csv files: only the first occurrence of each URL is kept.

As a dataset, we decided to scrape \url{corsi.unige.it} starting the procedure from \url{https://corsi.unige.it}. 
This decision was made for the following reasons:
\begin{itemize}
    \item it is monolingual (pages are in italian)
    \item we have prior knowledge of its structure 
    \item we know that it has recurring patterns in its content
\end{itemize}
In particular, we leverage the last point in Section MIAOMIAOMIAO to evaluate the quality of the inferred network.

Thanks to the scraping started on 3 June 2020, we obtained a dataset of 20722 documents and 29286 unique terms. 
After the preprocessing phase, the number of documents and words remain the same.
These results are consistent with our prior knowledge of the website and the assumtion we made before.



\section{Similarity graph} \label{sgexp}
After preprocessing, we performed Topic Modeling using HDP. 
In particular, in this phase the content of each HTML document is
parsed and only the raw text is kept to be used for inference. Common stop words are also removed.

Given the document-topic matrix,
we then computed the Hellinger distance (see Section \ref{hd}) between each pair of documents $(p, q)$:
\[\mathit{HD}(p, q) = \mathit{HD}(q, p) = \frac{1}{\sqrt{2}} \sqrt{\sum_i (\sqrt{p(i)} - \sqrt{q(i)})^2}\]
The final output is can be viewed as a similarity graph represented through an adjacency matrix.
Since each computed Hellinger distance $d$ is always in the interval $[0, 1]$ by definition, the distance 
$1-d$ can be viewed as the weight of the edge between two documents.

One problem that we must deal with is the fact that
webpages in a domain could have portions of HTML in common which are not relevant to the content of the page itself.
For this reason, we decided to keep two similarity graphs:
\begin{itemize}
    \item a filtered version where words that appear in more than 50\% of documents are not considered
    \item an unfiltered one
\end{itemize}



\section{Scaling issues} \label{scaling}

All source code is licensed under the GNU General Public License v3.0 and it can be found at \url{https://github.com/Davide95/msc_thesis/tree/master/code}.
It is written in Python 3.7 and the external libraries used are reported in the relative \texttt{requirements.txt} file.
The source code was also analyzed with \say{flake8} and \say{pylint} to improve readability and to spot bugs.

During the writing of the source code, we encountered some scalability issues related to
the size of the datasets and the computational complexity of the algorithms used.
This section contains a summary of the main strategies used to overcome this
problem. All experiments were tested on a PowerEdge T630 server with the following characteristics:
\begin{itemize}
    \item Intel(R) Xeon(R) CPU E5-2630 v3 @ 2.40GHz (2 sockets)
    \item 8x16GiB of RAM
\end{itemize}

\subsection{NumPy}
The majority of operations in our source code are applied to vectors and matrices.
Modern CPUs provide SIMD (Single Instruction, Multiple Data) instructions like
Intel\textsuperscript{®} AVX-512 to speed up this kind of computations.
In order to use them in the source code, we make use of a library called NumPy\footnote{\url{https://numpy.org/}}
which is interoperable between different hardware and computing platforms.

Furthermore, it allows us to have more numerical stability when dealing with huge
matrices and vectors since it takes into account the limitations
related to storing real numbers using finite-precision floating-point representations
(for instance, using pairwise summation in its source code at
\href{https://github.com/numpy/numpy/blob/v1.18.1/numpy/core/src/umath/loops.c.src}{numpy/core/src/umath/loops.c.src}).

The interested reader can find more advantages of using NumPy reading \cite{5725236}.

\subsection{Numba}
When vectorization is not enough due to the computational costs of the algorithms used, we
compile portions of the Python code to machine code through Numba\footnote{\url{https://numba.pydata.org/}}.

Thanks to this tool, portions of the software can:
\begin{itemize}
    \item run at native machine code speed
    \item run without the involvement of the Python interpreter
    \item be parallelized using multithreading
\end{itemize}
In particular, parallelization is done releasing first the GIL (Global Interpreter Lock).
The interested reader can find more details about it
at \url{https://docs.python.org/3/glossary.html\#term-global-interpreter-lock}.

A basic introduction to how Numba works can be found at \cite{10.1145/2833157.2833162}.

\subsection{Multiprocessing}
We are not able to use NumPy nor Numba where the code is not numerically orientated.
Furthermore, it is not possible to release the GIL in some portions of the code
due to the limitations of the imported libraries.
For these reasons, there are situations in which multiprocessing is used.

In particular, we have to use multiprocessing instead of multithreading during the parsing of
HTML documents explained in Section \ref{sgexp}.
This choice allows us to scale using every core of our machine
but at the cost of increasing the memory consumption of the application.
To have a trade-off between computational costs
and memory issues, we decided to add two parameters that can be tweaked.

\subsection{Optimizing memory usage}
In Python, it is not possible to deallocate manually memory blocks; a GC (Garbage Collector)
is instead responsible to remove data from memory when needed.

To help its work, we try to avoid using the Jupyter Notebook when we found high memory usage.
The reason is mainly that it tends to abuse the use of the global scope for storing variables
and it keeps additional references to objects. Furthermore, we split the code into different steps,
each step restrained in a function.
Between each step, we manually force the GC to collect unused resources.

Finally, we avoid using lists when they do not have a fixed size.
The reason is that frequent append operations in a short time lead in multiple reallocations of
memory which could not be deallocated in time and a memory shortage might happen.
In particular, the problem is that lists in Python are not linked lists but they use
exponential over-allocation every time they lack space. To make a comparison, they have the same
mechanism as ArrayList in Java.
For a more detailed explanation of how memory management works for lists in Python, please refer to
its source code at
\href{https://github.com/python/cpython/blob/3.7/Objects/listobject.c}{Objects/listobject.c}.
